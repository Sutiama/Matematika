%! Author = matya
%! Date = 24.03.2023

\newpage
\chapter{Výroková logika}
V této kapitole se budeme podrobněji zabývat výrokovou logikou, jejími základními pojmy, pravidly a pravdivostními tabulkami.
Zjistíme, jak výroky mohou být propojeny pomocí logických operací, jako jsou například konjunkce, disjunkce a negace, a jak
lze s nimi pracovat, aby se zjistila jejich pravdivost. Výroková logika tvoří základ pro mnoho dalších oblastí logiky a matematiky
a je nezbytná pro porozumění matematickým důkazům a vědeckým teoriím.
    \section{Základní pojmy}
Základem výrokové logiky výrok. Výrok může být cokoliv, o čem můžeme prohlásit, zda je pravdivé či nikoliv. Pokud je výrok pravdivý,
říkame, že jeho pravdivnostní hodnota je \textbf{TRUE}. Pokud pravdivý není, říkáme, že jeho pravdivnostní hodnota je \textbf{FALSE}.
Nejčastěji se tyto hodnoty kódují jako \textbf{TRUE} číslem \textbf{1} a \textbf{FALSE} číslem \textbf{0}
        \subsection{Disjunkce}
Disjunkce je pravdivá, pokud je alespoň jeden z výroků pravdivý. V programování se s ním můžeme setkat, jako \textbf{OR}. Vylučující nebo,
nebo také exkluzivní disjunkce je logická operace, která je pravdivá, pokud má výrok \emph{A} odlišnou pravdivnostní hodnotu než-li výrok
\emph{B}. V programování se s ním také můžeme setkat pod názvem \textbf{XOR}.

        \subsection{Konjunkce}

        \subsection{Negace }
Nejedná se o logickou spojku, jelikož při negaci pracujeme pouze s jediným výrokem.
        \subsection{Implikace}
        \subsection{Ekvivalence}

    \section{Matematický zápis}
    \section{Lineární funkce s absolutní hodnotou}
    \section{Lineární rovnice a nerovnice}
    \section{Lineární rovnice a nerovnice s absolutní hodnotou}


